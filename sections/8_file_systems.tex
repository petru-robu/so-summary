\section{File-System Interface}

A \textbf{file} is a contiguous logical address space defined by its creator, containing data (numeric, character, binary) or programs .

\subsection{File Attributes and Operations}

\textbf{File Attributes} are stored in the directory structure on disk and vary by system :
\begin{itemize}
    \item \textbf{Name:} Human-readable identifier.
    \item \textbf{Identifier:} Unique tag (number) within the file system.
    \item \textbf{Type:} Support for different formats (e.g., .exe, .txt).
    \item \textbf{Location:} Pointer to the device and location on that device.
    \item \textbf{Size:} Current file size.
    \item \textbf{Protection:} Controls for reading, writing, and executing.
    \item \textbf{Time, date, and user ID:} Data for security and usage monitoring.
\end{itemize}


\subsubsection{File Operations}
The operating system provides several basic system calls to manage files :
\begin{itemize}
    \item \textbf{Create / Delete / Truncate}
    \item \textbf{Read / Write:} Handled at specific pointer locations.
    \item \textbf{Seek:} Repositioning the pointer within the file.
    \item \textbf{Open($F_i$):} Searches the directory for entry $F_i$ and copies it to memory.
    \item \textbf{Close($F_i$):} Moves memory content back to the directory structure.
\end{itemize}

\subsubsection{File Locking}
Used to mediate access to a file, similar to reader-writer locks :
\begin{itemize}
    \item \textbf{Shared Lock:} Multiple processes can acquire concurrently (reader lock).
    \item \textbf{Exclusive Lock:} Only one process can hold the lock (writer lock).
    \item \textbf{Advisory vs. Mandatory:} In advisory, processes check lock status; in mandatory, the OS enforces access denial.
\end{itemize}

\subsection{Open Files}
To manage open files, the OS maintains an \textbf{open-file table} . Key data pieces include:
\begin{itemize}
    \item \textbf{File Pointer:} Per-process tracking of the last read/write location.
    \item \textbf{File-open Count:} Tracks how many processes have the file open to determine when to remove it from the table.
    \item \textbf{Disk Location:} Cached information for fast data access.
    \item \textbf{Access Rights:} Mode information (read/write) for each process.
\end{itemize}

\subsection{Directory Structure and Operations}
The directory is a collection of nodes containing information about all files .

\includegraphics[scale=0.22]{13.7.png} % Directory Structure Diagram

\textbf{Operations performed on directories:} Search, Create, Delete, List, Rename, and Traverse the system .

\subsection{Access Methods}

\subsubsection{Sequential Access}
Information is processed in order, one record after the other .
\begin{itemize}
    \item \textbf{Operations:} \texttt{read next}, \texttt{write next}, \texttt{reset}.
\end{itemize}
\includegraphics[scale=0.25]{13.16.png} % Sequential Access Diagram

\subsubsection{Direct Access}
Allows fixed-length logical records to be read or written in any order by referring to a \textbf{relative block number} ($n$) .
\begin{itemize}
    \item \textbf{Operations:} \texttt{read n}, \texttt{write n}, \texttt{position to n}.
\end{itemize}

\subsubsection{Simulating Sequential on Direct Access}
Sequential access can be simulated by maintaining a current position pointer ($cp$) :
\begin{itemize}
    \item \textbf{Reset:} $cp = 0;$
    \item \textbf{Read Next:} \texttt{read cp; cp = cp + 1;}
\end{itemize}

\subsection{Disk and File System Organization}
Disks are subdivided into \textbf{partitions} (or slices) .

\includegraphics[scale=0.2]{13.22.png} % Typical File-System Organization

\subsection{Directory Logical Structures}

\subsubsection{Single-Level Directory}
All files are contained in the same directory for all users .
\begin{itemize}
    \item \textbf{Pros:} Simple.
    \item \textbf{Cons:} Naming conflicts (two users can't use the same name) and no grouping.
\end{itemize}
\includegraphics[scale=0.2]{13.27.png} % Single-Level Diagram

\subsubsection{Two-Level Directory}
Each user has their own \textbf{user file directory} (UFD) under a \textbf{master file directory} (MFD) .
\begin{itemize}
    \item \textbf{Pros:} Solves naming conflicts; efficient searching via path names.
    \item \textbf{Cons:} Still lacks sophisticated grouping capability.
\end{itemize}
\includegraphics[scale=0.2]{13.28.png} % Two-Level Diagram

\subsubsection{Tree-Structured Directories}
The most common structure, allowing users to create subdirectories and organize files arbitrarily .
\includegraphics[scale=0.2]{13.29.png} % Tree-Structured Diagram

\subsubsection{Acyclic-Graph Directories}
Allows shared subdirectories and files, meaning a file can exist in two directories simultaneously (aliasing) .
\begin{itemize}
    \item \textbf{Implementation:} Uses \textbf{Links} (pointers to existing files).
    \item \textbf{Deletion:} Can lead to dangling pointers. Solutions include \textbf{backpointers} or \textbf{reference counts} (only delete file when count = 0).
\end{itemize}
\includegraphics[scale=0.2]{13.30.png} % Acyclic-Graph Diagram

\subsubsection{General Graph Directory}
Allows cycles within the directory structure .
\begin{itemize}
    \item \textbf{Cycle Detection:} To avoid infinite loops during traversal, systems may use \textbf{garbage collection} or a \textbf{cycle detection algorithm} every time a link is added.
\end{itemize}
\includegraphics[scale=0.2]{13.32.png} % General Graph Diagram

\subsection{Protection}
The file owner must control what can be done and by whom .
\textbf{Access Types:} Read, Write, Execute, Append, Delete, List .

\subsubsection{UNIX RWX Access}
In UNIX/Linux, protection is handled via three classes of users with three bits (Read: first bit, Write: second bit, eXecute: third bit) :
\begin{enumerate}
    \item \textbf{Owner:} (e.g., 7 $\rightarrow$ 111)
    \item \textbf{Group:} (e.g., 6 $\rightarrow$ 110)
    \item \textbf{Public:} (e.g., 1 $\rightarrow$ 001)
\end{enumerate}
Commands like \texttt{chmod 761 game} define these permissions, while \texttt{chgrp} attaches a specific group to a file .